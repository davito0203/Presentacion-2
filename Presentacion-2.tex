\documentclass[pstricks, 12pt]{beamer}

\usetheme{JuanLesPins}
\useinnertheme{rounded}
\useoutertheme{infolines}
%\usecolortheme{fly}
\usefonttheme[onlymath, stillsansserifmath]{serif}
\usefonttheme{professionalfonts} %nuevo

\usepackage[utf8]{inputenc}
\usepackage{amsthm}
\usepackage[spanish]{babel}
\usepackage{alltt}
\usepackage{colortbl}
\usepackage{enumerate}
\usepackage{pdfpages}
\usepackage{multimedia}
\setbeamercovered{transparent}
\usepackage{tabulary}
\usepackage{graphicx}

\title[Habilitador]{Habilitador ora para personas con discapacidades auditivas.}

\author[Martinez, David]{David Ricardo Martinez Hernandez}
\institute[UNAL]{\large Universidad Nacional de Colombia}
\date[12/02/14]{Diciembre 3, 2014}

\begin{document}
\everymath{\displaystyle}
	
\begin{frame}
  \titlepage
\end{frame}

\begin{frame}{Tabla de contenido}
  \tableofcontents
\end{frame}

\section{Objetivos}
\subsection{Objetivo General}
\begin{frame}{Objetivo general}
 \begin{itemize}
  \item Desarrollar una herramienta para asistir la habilitación en personas con discapacidad auditiva.
 \end{itemize}
\end{frame}

\subsection{Objetivos Específicos}
\begin{frame}{Objetivos específicos}
 \begin{itemize}
  \item Identificar con la ayuda de un equipo de profesionales del lenguaje los aspectos de la metodología de oralización que son susceptibles a asistir tecnológicamente desde su punto de vista.
  \item Diseñar conceptualmente la herramienta en un lenguaje de alto nivel que brinde la oportunidad de validar y verificar el sistema desde el punto de vista funcional.
  \item Determinar la partición entre hardware y software que mejor se adapte al diseño conceptual de la herramienta.
 \end{itemize}
\end{frame}

\section{Metodología}
\begin{frame}{Metodología}
 \begin{itemize}
  \item Estudio del estado del arte
  \item Identificación de los aspectos de la metodología de oralización.
  \item Diseño conceptual de la herramienta en un lenguaje de alto nivel.
  \item Determinar la partición entre hardware y software que mejor se adapte al diseño.
  \item Diseño del algoritmo que permita la validación y verificación del sistema.
  \item Diseño de la interfaz gráfica que permita la mejor comprensión para su utilización y validación.
  \item Desarrollo de Handbook del usuario y desarrollador.
 \end{itemize}
\end{frame}

\section{Pasos del libro}

\begin{frame}{Nombre}
 \begin{itemize}
  \item No se ha pensado en un nombre y tampoco en un icono para la aplicación.
 \end{itemize}
\end{frame}

\begin{frame}{Misión y Visión}
 
Misión: 
 \begin{itemize}
  \item Ayudar a habilitar a personas sordas de nacimiento para que pueden interactuar con nosotros.
  \item Ayudar a la rehabilitación de personas que han sufrido o tienen discapacidades auditivas
 \end{itemize}

Visión:
\begin{itemize}
 \item No se ha pensado en un visión.
\end{itemize}
\end{frame}

\begin{frame}{Licencia}
 \begin{itemize}
  \item En principio la licencia que tendría el software sería una GPLv3 o GPLv2.
  \item No se ha definido una licencia en particular porque no se sabe que políticas tiene la universidad con los desarrollos de software para las tesis.
 \end{itemize}
\end{frame}

\begin{frame}{Características y Lista de requerimientos}
 \begin{itemize}
  \item Compiladores y librerías para Android.
  \item Hardware compatible con Android.
  \item Análisis de la voz.
  \item Identificación de la entonación para:
  \begin{itemize}
   \item Preguntas.
   \item Afirmación.
  \end{itemize}
 \end{itemize}
\end{frame}

\begin{frame}{Estado de desarrollo}
 \begin{itemize}
  \item Ya se ha realizado la investigación del estado del arte.
  \item Se encuentra actualmente en una fase de pruebas de software y hardware.
 \end{itemize}
\end{frame}

\begin{frame}{Descargable}
 \begin{itemize}
  \item No se tiene ninguna versión.
  \item Cuando se tenga la primera versión será descargable el codigo fuente y la aplicación para el dispositivo.
 \end{itemize}
\end{frame}

\begin{frame}{Acceso al control de versiones y Bug tracker}
 \begin{itemize}
  \item Esta tarea se realizara por medio de GitHub.
 \end{itemize}
\end{frame}

\begin{frame}{Canales de comunicación}
 \begin{itemize}
  \item Se utilizará el blog o la pagina que se realizó en el curso.
  \item Permitirá la interacción de los usuarios con los desarrolladores.
  \item En la pagina se encontrara siempre las versiones que han sido desarrolladas.
 \end{itemize}
\end{frame}

\begin{frame}{Guías para desarrolladores}
 \begin{itemize}
  \item No se ha desarrollado aún las guías para los desarrolladores.
  \item No se ha desarrollado aún las guías para los usuarios finales.
 \end{itemize}
\end{frame}

\begin{frame}{Documentación}
 \begin{itemize}
  \item Se tendrá toda la documentación necesaria  en la pagina.
  \item Como descargar el código.
  \item Realizar las contribuciones.
  \item Actualizar las mejoras realizadas.
  \item Desarrollos similares:
  \begin{itemize}
   \item Praat.
   \item Spectral Audio Analyzer.
   \item Pauper Tango.
   \item TacaID.
   \item Implantes Cocleares
  \end{itemize}
 \end{itemize}
\end{frame}

\begin{frame}{Ejemplo de salida y Screenshots}
 \begin{itemize}
  \item No se tiene aún nada de los ejemplos de salida, estos se actualizaran cuando la aplicación tenga un avance significativo.
 \end{itemize}
\end{frame}

\begin{frame}{Hosting}
 \begin{itemize}
  \item Se utilizará GitHub para esta tarea.
  \item También la pagina web.
 \end{itemize}
\end{frame}

\end{document}